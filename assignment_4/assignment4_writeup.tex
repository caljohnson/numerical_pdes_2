\documentclass[12pt]{article}
\textwidth=17cm \oddsidemargin=-0.9cm \evensidemargin=-0.9cm
\textheight=23.7cm \topmargin=-1.7cm
\headheight=14.5pt

\usepackage{amssymb, amsmath, amsfonts}
\usepackage{moreverb}
\usepackage{graphicx}
\usepackage{enumerate}
\usepackage{graphics}
\usepackage{color}
\usepackage{array}
\usepackage{float}
\usepackage{hyperref}
\usepackage{textcomp}
\usepackage{alltt}
\usepackage{physics}
\usepackage{mathtools}
\usepackage{tikz}
\usetikzlibrary{positioning}
\usetikzlibrary{arrows}
\usepackage{bigints}
\usepackage[utf8]{inputenc}
\usepackage[english]{babel}
\usepackage{amsthm}
\usepackage{fancyhdr}
\usepackage[makeroom]{cancel}
\pagestyle{fancy}
\allowdisplaybreaks

\newcommand{\E}{\varepsilon}

\newcommand{\suchthat}{\, \mid \,}
\newcommand{\ol}[1]{\overline{#1}}
\newcommand{\bbar}[1]{\overline{#1}}
\newcommand{\inpd}[1]{{\left< \, #1 \, \right>}}
\renewcommand{\theenumi}{\alph{enumi}}
\newcommand\Wider[2][3em]{%
\makebox[\linewidth][c]{%
  \begin{minipage}{\dimexpr\textwidth+#1\relax}
  \raggedright#2
  \end{minipage}%
  }%
}

\def\R{\mathbb{R}}
\def\C{\mathbb{C}}
\def\H{\mathcal{H}}
\DeclareMathOperator*{\esssup}{\text{ess~sup}}
\newcommand{\resolv}[1]{\rho(#1)}
\newcommand{\spec}[1]{\sigma(#1)}
\newcommand{\iffR}{\noindent \underline{$\Longrightarrow$:} }
\newcommand{\iffL}{\noindent \underline{$\Longleftarrow$:} }
\newcommand{\lightning}{\textbf{\Huge \Lightning}}
\newcommand{\spt}[1]{\text{spt}(#1)}
\def\ran{\text{ ran}}
   
\newenvironment{myprob}[1]
    {%before text commands
    %{\Huge \_ \_ \_ \_ \_ \_ \_ \_ \_ \_ \_ \_ \_ \_ \_ \_ \_ \_ } \\
    \noindent{\Huge$\ulcorner$}\textbf{#1.}\begin{em}
    }
    { 
    %after text commands
    \end{em} \\ \hphantom{l} \hfill {\Huge$\lrcorner$} }
%	{\noindent \rule{7.5cm}{2pt} \textgoth{#1} \rule{8.cm}{2pt} \begin{em}}
%	{\end{em}\\ \vspace{0.1pt}\noindent \rule{\textwidth}{2pt}}
%
\setcounter{section}{-1}




\begin{document}
\lhead{MATH228B}
\chead{Carter Johnson - Homework 04}
\rhead{\today}

{\let\newpage\relax} 


%%%%%%%%%%%%%%%%%%%%%%%%%%%%%%%%%%%%%%%%%%%%%%%%%%%%% P2
\begin{myprob}{Problem 1}
Write programs to solve the advection equation
$$u_t + au_x = 0, $$
on $[0,1]$ with periodic boundary conditions using upwinding and Lax-Wendroff.  For smooth solutions, we expect upwinding to be first-order accurate and Lax-Wendroff to be second-order accurate, but it is not clear what accuracy to expect for nonsmooth solutions.
\end{myprob}
\begin{enumerate}[(a)]
\item Let $a=1$ and solve the problem up to time $t=1$.  Perform a refinement study for both upwinding and Lax-Wendroff with $\Delta t=0.9a\Delta x$ with a smooth initial condition.  Compute the rate of convergence in the 1-norm, 2-norm, and max-norm.  Note that the exact solution at time $t=1$ is the initial condition, and so computing the error is easy.


\item Repeat the previous problem with the discontinuous initial condition
$$u(x,0) = \begin{cases}1 & \text{ if } |x-1/2|<1/4 \\ 0 & \text{ otherwise } \end{cases}.$$


\end{enumerate}

\begin{myprob}{Problem 2}
For solving the heat equation we frequently use Crank-Nicolson, which is trapezoidal rule time integration with a second-order space discretization.  The analogous scheme for the linear advection equation is
$$u_{j+1}^{n+1} - u_{j}^{n} + \dfrac{\nu}{4}\qty(u_{j+1}^{n}-u_{j-1}^{n}) + \dfrac{\nu}{4}\qty(u_{j+1}^{n+1}-u_{j-1}^{n+1}) =0,$$
where $\nu = a\Delta t/\Delta x$.
\end{myprob}
\begin{enumerate}[(a)]
\item Use von Neumann analysis to show that this scheme is unconditionally stable and that $\norm{u^n}_2 = \norm{u^0}_2$.  This scheme is said to be nondissipative- i.e., there is no amplitude error.  This seems reasonable because this is a property of the PDE.

\item Solve the advection equation on the periodic domain $[0,1]$ with the initial condition from problem 1(b).  Show the solution and comment on your results.

\item Compute the relative phase error as arg($g(\theta)$)/($-\nu\theta$), where $g$ is the amplification factor and $\theta = \xi\Delta x$, and plot it for $\theta\in[0,\pi]$.  How does the relative phase error and lack of amplitude error relate to the numerical solutions you observed in part (b).
\end{enumerate}


\end{document}